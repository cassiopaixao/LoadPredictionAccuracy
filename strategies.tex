\section{Server Consolidation Strategies}
\label{sec:server_consolidation_strategies}

This section describes server consolidation strategies that are evaluated in this paper. We also discuss possible effects of inaccurate load predictions.

\subsection{Basic strategies}

We call basic strategies those heuristics that are usually considered to solve bin packing problems. Given a set of virtual machines and a set of available physical machines:

\begin{itemize}
	\item \textbf{First-Fit:}
		each virtual machine is allocated to the first physical machine that is able to provide the resources demanded by the virtual machine;
		
	\item \textbf{Best-Fit:}
		each virtual machine is allocated to the physical machine that can meet the virtual machine's demand and whose allocation results in the smallest residual capacity; and

	\item \textbf{Worst-Fit:}
		each virtual machine is allocated to the physical machine that can meet the virtual machine's demand and whose allocation results in the biggest residual capacity.
\end{itemize}

The algorithms search first for an already used physical machine, but if none of them could host the virtual machine, so one unused physical machine is selected (by the same constraints) to be ``powered up'' and to host the virtual machine.

In the simulations, we used the ``decreasing'' version of these strategies, i. e., First-Fit Decreasing, Best-Fit Decreasing and Worst-Fit Decreasing. In these versions, one step is added before the ordinary algorithm be executed: the virtual machines are decreasingly ordered (in a lexicographical order).

\subsubsection*{Migration Control Strategies}

While the virtual machines are running, their resources' demands can change or not. A migration control strategy is used to decide when a virtual machine should be migrated.
Ferreto et al.~\cite{ferreto2011server} proposed a migration control strategy based on virtual machines' demands changes: if a virtual machine's demands change, so the virtual machine should be migrated.
We added one more constraint in Ferreto et al.'s strategy and evaluated the following situation too: if a virtual machine's demands change in a way that can overload the physical machine, so the virtual machine should be migrated.


\subsection{Khanna et al.~\cite{khanna2006application} strategy}

Khanna et al. proposed algorithms trying ``\emph{to resolve SLA violations by reallocating VMs to PMs, as needed}''.
The proposed algorithm is reactive and is triggered to achieve two objectives: to resolve (or avoid) SLA violations and to turn off machines with low utilization.

Before explaining the algorithm, we have to set some terms.
Each virtual machine (VM) was represented as a $d$-dimensional vector where each dimension represents one of the resources, and the resource utilization of $VM_i$ at time $t$ is represented by $U_i(t)$.
Each physical machine, $PM_j$, has a fixed capacity $C_j$ in a $d$-dimensional space. The $L_j(t)$ vector is the sum of $U_k(t)$, for all $VM_k$ allocated in the $PM_j$ at time $t$, and $R_j(t)$ is the residual capacity vector of the $j^{th}$ physical machine at time $t$. 
In each $i^{th}$ dimension, the associated resource has a \emph{knee} value $n_i$, chosen as a superior limit of this resource utilization (if the utilization of this resource in one physical machine extrapolates this value, probably a SLA violation will occur).

% need to rewrite this. It's exactaly how is written in the original paper. Need to explain the discrete time instances?
! This dynamic algorithm runs at discrete time instances $t_0, t_1, \ldots$  to perform reallocation when triggered via a resource threshold violation alert. !

An initial allocation state is provided, and the $d$ resources are monitored for each virtual and physical machine. The algorithm to migrate VMs from $PM_i$ is triggered when a resource utilization in the physical machine $PM_i$ reaches its \emph{knee} value, or when the $PM_i$ utilization falls below a low mark (all VMs allocated there are migrated in this case).

\begin{enumerate}
	\item ! Select the VM from $PM_i$ with the lowest utilization and move it to a physical machine which has the least residual capacity big enough to hold this VM !;
	\begin{enumerate}
		\item If no physical machine has big enough residual capacity to hold this VM, then instantiate a new physical machine and allocate the VM to that machine.
	\end{enumerate}
	\item If the SLA constraints aren't satisfied yet, repeat the first step until the constraints are satisfied;
\end{enumerate}

No strategy was showed to allocate new virtual machines, so we decided to use the same strategy presented to choose whose physical machine would host a virtual machine that has to be migrated.


\subsection{Strategy 3}
